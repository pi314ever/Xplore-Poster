% Gemini theme
% See: https://rev.cs.uchicago.edu/k4rtik/gemini-uccs
% A fork of https://github.com/anishathalye/gemini

\documentclass[final]{beamer}

% ====================
% Packages
% ====================

\usepackage[T1]{fontenc}
\usepackage{lmodern}
\usepackage{anyfontsize}
\usepackage[size=custom,width=120,height=72,scale=1.0]{beamerposter}
\usetheme{gemini}
% \usecolortheme{uchicago}
\usecolortheme{stanford}
\usepackage{graphicx}
\usepackage{booktabs}
\usepackage{tikz}
\usepackage{pgfplots}
\pgfplotsset{compat=1.17}

% ====================
% Lengths
% ====================

% If you have N columns, choose \sepwidth and \colwidth such that
% (N+1)*\sepwidth + N*\colwidth = \paperwidth
\newlength{\sepwidth}
\newlength{\colwidth}
\setlength{\sepwidth}{0.025\paperwidth}
\setlength{\colwidth}{0.3\paperwidth}

\newcommand{\separatorcolumn}{\begin{column}{\sepwidth}\end{column}}


\title{Real-Time Lens Distortion with Deep Learning}


\author{Daniel Huang\inst{1}}

\institute[shortinst]{\inst{1} Institute of Computational and Mathematical Engineering, Stanford}

% ====================
% Footer (optional)
% ====================

\footercontent{
  \href{https://github.com/pi314ever/mathworks-physical-sensor-model}{GitHub: pi314ever/mathworks-physical-sensor-model} \hfill
  ICME Xpo Research Symposium 2023\hfill
  \href{mailto:dhuangpi@alumni.stanford.edu}{dhuangpi@alumni.stanford.edu}}
% (can be left out to remove footer)

% ====================
% Logo (optional)
% ====================

% use this to include logos on the left and/or right side of the header:
% \logoright{\includegraphics[height=7cm]{logos/cs-logo-maroon.png}}
% \logoleft{\includegraphics[height=7cm]{logos/cs-logo-maroon.png}}

% ====================
% Body
% ====================

\begin{document}

% This adds the Logos on the top left and top right
\addtobeamertemplate{headline}{}
{
    \begin{tikzpicture}[remember picture,overlay]
      \node [anchor=north west, inner sep=3cm] at ([xshift=0.0cm,yshift=1.0cm]current page.north west)
      {\includegraphics[height=4.0cm]{stanford_logos/icme_logo.png}};
      \node [anchor=north east, inner sep=3cm] at ([xshift=0.0cm,yshift=2.5cm]current page.north east)
      {\includegraphics[height=7.0cm]{stanford_logos/Block_S_2_color.png}};
    \end{tikzpicture}
}

\begin{frame}[t]
\begin{columns}[t]
\separatorcolumn

\begin{column}{\colwidth}

  \begin{block}{Introduction}

    % TODO: Fix up rough draft
    \begin{itemize}
      \item Motivation: What is the big scope problem?\\
      Training self-driving cars requires a lot of data, but it is expensive/time-consuming/unsafe to collect data in real-world environments.\\
      \item What is the specific problem?\\
      The world data can be generated through virtual worlds (e.g. Unity, Unreal Engine), but the virtual world data needs to have accurate lens distortion properties for the specific cameras used in self-driving cars.
    \end{itemize}

  \end{block}

  \begin{block}{Background: Brown's Distortion Model}
    The Brown's distortion model describes radial and tangential lens distortion given calibrated camera parameters using an infinite polynomial series. The radial model is given by:
    \begin{align*}
      x_{d,r}&= x + \bar{x} \left( k_1 r^2 + k_2 r^4 + k_3 r^6 + ... \right) \\
      y_{d,r}&= y + \bar{y} \left( k_1 r^2 + k_2 r^4 + k_3 r^6 + ... \right) \\
    \end{align*}
    And the tangential model is given by:
    \begin{align*}
      x_{d,t}&= x + \left[p_1 (r^2 + 2 \bar{x}^2) + 2 p_2 \bar{x} \bar{y}\right] \left(1 + p_3 r^2 + p_4 r^4 + ...\right)\\
      y_{d,t}&= y + \left[p_2 (r^2 + 2 \bar{y}^2) + 2 p_1 \bar{x} \bar{y}\right]\left(1 + p_3 r^2 + p_4 r^4 + ...\right)
    \end{align*}
    where $x, y$ are the undistorted coordinates, $x_{d,*}, y_{d,*}$ are the distorted coordinates, $\bar{x}, \bar{y}$ are distortion-centered coordinates, $r^2 = \bar{x}^2 + \bar{y}^2$, $k_i$ are the radial distortion parameters, and $p_i$ are the tangential distortion parameters.
  \end{block}

  \begin{block}{Background: Previous Approaches}

  \end{block}

  \begin{alertblock}{A highlighted block}

  \end{alertblock}

\end{column}

\separatorcolumn

\begin{column}{\colwidth}

  \begin{block}{Methodology}

  \end{block}

  \begin{block}{Training Flow}

  \end{block}

\end{column}

\separatorcolumn

\begin{column}{\colwidth}

  \begin{block}{Results}

  \end{block}

  \begin{block}{References}

    \nocite{*}
    \footnotesize{\bibliographystyle{plain}\bibliography{poster}}

  \end{block}

\end{column}

\separatorcolumn
\end{columns}
\end{frame}

\end{document}
